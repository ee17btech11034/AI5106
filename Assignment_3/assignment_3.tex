\documentclass[journal,12pt,twocolumn]{IEEEtran}
%
\usepackage{setspace}
\usepackage{gensymb}
%\doublespacing
\singlespacing

\usepackage{graphicx}
\usepackage[cmex10]{amsmath}
\usepackage{amsmath,amsthm}
\usepackage{mathrsfs}
\usepackage{txfonts}
\usepackage{stfloats}
\usepackage{bm}
\usepackage{cite}
\usepackage{cases}
\usepackage{subfig}

\usepackage{longtable}
\usepackage{multirow}
\usepackage{commath}
\usepackage{enumitem}
\usepackage{mathtools}
\usepackage{steinmetz}
\usepackage{tikz}
\usepackage{circuitikz}
\usepackage{verbatim}
\usepackage{tfrupee}
\usepackage[breaklinks=true]{hyperref}

\usepackage{tkz-euclide}

\usetikzlibrary{calc,math}
\usepackage{listings}
    \usepackage{color}                                            
    \usepackage{array}                                            
    \usepackage{longtable}                                        
    \usepackage{calc}                                             
    \usepackage{multirow}                                         
    \usepackage{hhline}                                           
    \usepackage{ifthen}                                           
    \usepackage{lscape}     
\usepackage{multicol}
\usepackage{chngcntr}

\DeclareMathOperator*{\Res}{Res}

\renewcommand\thesection{\arabic{section}}
\renewcommand\thesubsection{\thesection.\arabic{subsection}}
\renewcommand\thesubsubsection{\thesubsection.\arabic{subsubsection}}

\renewcommand\thesectiondis{\arabic{section}}
\renewcommand\thesubsectiondis{\thesectiondis.\arabic{subsection}}
\renewcommand\thesubsubsectiondis{\thesubsectiondis.\arabic{subsubsection}}

\hyphenation{op-tical net-works semi-conduc-tor}
\def\inputGnumericTable{}                                 

\lstset{
%language=C,
frame=single, 
breaklines=true,
columns=fullflexible
}
\lstset{
%language=TeX,
frame=single, 
breaklines=true
}

\begin{document}


\newtheorem{theorem}{Theorem}[section]
\newtheorem{problem}{Problem}
\newtheorem{proposition}{Proposition}[section]
\newtheorem{lemma}{Lemma}[section]
\newtheorem{corollary}[theorem]{Corollary}
\newtheorem{example}{Example}[section]
\newtheorem{definition}[problem]{Definition}

\newcommand{\BEQA}{\begin{eqnarray}}
\newcommand{\EEQA}{\end{eqnarray}}
\newcommand{\define}{\stackrel{\triangle}{=}}
\bibliographystyle{IEEEtran}
\providecommand{\mbf}{\mathbf}
\providecommand{\pr}[1]{\ensuremath{\Pr\left(#1\right)}}
\providecommand{\qfunc}[1]{\ensuremath{Q\left(#1\right)}}
\providecommand{\sbrak}[1]{\ensuremath{{}\left[#1\right]}}
\providecommand{\lsbrak}[1]{\ensuremath{{}\left[#1\right.}}
\providecommand{\rsbrak}[1]{\ensuremath{{}\left.#1\right]}}
\providecommand{\brak}[1]{\ensuremath{\left(#1\right)}}
\providecommand{\lbrak}[1]{\ensuremath{\left(#1\right.}}
\providecommand{\rbrak}[1]{\ensuremath{\left.#1\right)}}
\providecommand{\cbrak}[1]{\ensuremath{\left\{#1\right\}}}
\providecommand{\lcbrak}[1]{\ensuremath{\left\{#1\right.}}
\providecommand{\rcbrak}[1]{\ensuremath{\left.#1\right\}}}
\theoremstyle{remark}
\newtheorem{rem}{Remark}
\newcommand{\sgn}{\mathop{\mathrm{sgn}}}
\providecommand{\abs}[1]{\(\left\vert#1\right\vert\)}
\providecommand{\res}[1]{\Res\displaylimits_{#1}} 
\providecommand{\norm}[1]{\(\left\lVert#1\right\rVert\)}
%\providecommand{\norm}[1]{\lVert#1\rVert}
\providecommand{\mtx}[1]{\mathbf{#1}}
\providecommand{\mean}[1]{E\(\left[ #1 \right]\)}
\providecommand{\fourier}{\overset{\mathcal{F}}{ \rightleftharpoons}}
%\providecommand{\hilbert}{\overset{\mathcal{H}}{ \rightleftharpoons}}
\providecommand{\system}{\overset{\mathcal{H}}{ \longleftrightarrow}}
	%\newcommand{\solution}[2]{\textbf{Solution:}{#1}}
\newcommand{\solution}{\noindent \textbf{Solution: }}
\newcommand{\cosec}{\,\text{cosec}\,}
\providecommand{\dec}[2]{\ensuremath{\overset{#1}{\underset{#2}{\gtrless}}}}
\newcommand{\myvec}[1]{\ensuremath{\begin{pmatrix}#1\end{pmatrix}}}
\newcommand{\mydet}[1]{\ensuremath{\begin{vmatrix}#1\end{vmatrix}}}
%\numberwithin{equation}{section}
\numberwithin{equation}{subsection}
%\numberwithin{problem}{section}
%\numberwithin{definition}{section}
\makeatletter
\@addtoreset{figure}{problem}
\makeatother
\let\StandardTheFigure\thefigure
\let\vec\mathbf
%\renewcommand{\thefigure}{\theproblem.\arabic{figure}}
\renewcommand{\thefigure}{\theproblem}
%\setlist[enumerate,1]{before=\renewcommand\theequation{\theenumi.\arabic{equation}}
%\counterwithin{equation}{enumi}
%\renewcommand{\theequation}{\arabic{subsection}.\arabic{equation}}
\def\putbox#1#2#3{\makebox[0in][l]{\makebox[#1][l]{}\raisebox{\baselineskip}[0in][0in]{\raisebox{#2}[0in][0in]{#3}}}}
     \def\rightbox#1{\makebox[0in][r]{#1}}
     \def\centbox#1{\makebox[0in]{#1}}
     \def\topbox#1{\raisebox{-\baselineskip}[0in][0in]{#1}}
     \def\midbox#1{\raisebox{-0.5\baselineskip}[0in][0in]{#1}}
\vspace{3cm}
\title{Assignment 3}
\author{Raja Asiwal}
\maketitle
\newpage
%\tableofcontents
\bigskip
\renewcommand{\thefigure}{\theenumi}
\renewcommand{\thetable}{\theenumi}
\begin{abstract}
This document explains the concept of the equation of a circle passing through three points.
\end{abstract}
Download the python code from 
%
\begin{lstlisting}
https://github.com/ee17btech11034/AI5106/blob/main/Assignment_3/AI_assignment_3.py
\end{lstlisting}
%
and latex-tikz codes from 
%
\begin{lstlisting}
https://github.com/ee17btech11034/AI5106/blob/main/Assignment_3/assignment_3.tex
\end{lstlisting}
%
\section{Problem}
Find the equation of a circle that passes through the points  \begin{pmatrix} 1  \\ 2 \end{pmatrix}, \begin{pmatrix} 3 \\ -4 \end{pmatrix} and \begin{pmatrix} 5 \\ -6 \end{pmatrix}.

\section{Explanation}
General equation of circle is given by:\\
\begin{align}
    \vec{x}^T \vec{x} -2 \vec{c}^T \vec{x} + f = 0 
\end{align}   
$\vec{c}$ is the centre. Circle is passing through points $\vec{P}, \vec{Q}, \vec{R}$, so substituting these in circle equation
\begin{align}
    2 \begin{pmatrix} 1 & 2 \end{pmatrix} \vec{c} -f &= 5 \\
    2 \begin{pmatrix} 3 & -4 \end{pmatrix} \vec{c} -f &= 25 \\
    2 \begin{pmatrix} 5 & -6 \end{pmatrix} \vec{c} -f &= 61
\end{align}
can be expressed as 
\begin{align}
    \begin{pmatrix} 2 & 4 & -1 \\ 6 & -8 & -1 \\ 10 & -12 & -1 \end{pmatrix} \begin{pmatrix} \vec{c} \\ f \end{pmatrix} = \begin{pmatrix} 5 \\ 25 \\ 61 \end{pmatrix}
\end{align}

Row reducing the augmented matrix
\begin{align*}
    \begin{pmatrix} 2 & 4 & -1 & 5 \\ 6 & -8 & -1 & 25 \\ 10 & -12 & -1 & 61  \end{pmatrix} 
     \begin{array}{c}
     \genfrac{}{}{0pt}{0}{R_2 \rightarrow R_2 - 3 R_1}{R_3 \rightarrow R_3 - 5 R_1} \\  \longleftrightarrow
     \end{array}
    \begin{pmatrix} 2 & 4 & -1 & 5 \\ 0 & -20 & 2 & 10 \\ 0 & -32 & 4 & 36  \end{pmatrix} \\
    \begin{array}{c}
     \genfrac{}{}{0pt}{0}{R_1 \rightarrow R_1 + \frac{1}{5} R_2}{R_3 \rightarrow R_3 - \frac{8}{5} R_2} \\  \longleftrightarrow
     \end{array}
    \begin{pmatrix} 2 & 0 & -\frac{3}{5} & 7 \\ 0 & -20 & 2 & 10 \\ 0 & 0 & \frac{4}{5} & 20  \end{pmatrix} \\
    \begin{array}{c}
     \genfrac{}{}{0pt}{0}{R_1 \rightarrow R_1 + \frac{3}{4} R_3}{R_2 \rightarrow R_2 - \frac{5}{2} R_3} \\  \longleftrightarrow
     \end{array}
    \begin{pmatrix} 2 & 0 & 0 & 22 \\ 0 & -20 & 0 & -40 \\ 0 & 0 & \frac{4}{5} & 20  \end{pmatrix}
\end{align*}
\section{Solution}
So, the centre of circle is :
\begin{align}
   \vec{c} = \begin{pmatrix}  \frac{22}{2} \\ \frac{-40}{-20} \end{pmatrix} = \begin{pmatrix}  11 \\ 2 \end{pmatrix}  \\
    f = \frac{20}{\frac{4}{5}} = 25 
\end{align}
Equation of circle is:
\begin{align}
    \vec{x}^T \vec{x} -2\begin{pmatrix}  11 & 2 \end{pmatrix} \vec{x} + 25 = 0 
\end{align}
\begin{figure}[t]
    \centering
    \includegraphics[width = \columnwidth]{AI_assignment_3.png}
    \caption{Circle passing through three points}
    \label{fig:Circle}
\end{figure}

\end{document}
